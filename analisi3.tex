\documentclass[a4paper,portrait,columns=3,5pt]{cheatsheet}
\title{Formulario Analisi 3}
%\author{Elena Acinapura}
\author{}
\date{\today}

\usepackage{commath}
\usepackage{graphicx}
\usepackage{mathtools}
\usepackage{commath}
\usepackage{amssymb}
\usepackage{amsmath}
\usepackage{bigints}
\usepackage[utf8]{inputenc}

\begin{document}
%\maketitle

\section{Formule trigonometriche}
\subsection{Addizione e sottrazione}
$$\sin (\alpha + \beta) = \sin(\alpha) \cos(\beta) + \cos(\alpha)\sin(\beta)$$
$$\sin(\alpha - \beta) = \sin(\alpha) \cos(\beta) - \cos(\alpha)\sin(\beta)$$
$$\cos(\alpha + \beta) = \cos(\alpha) \cos(\beta) - \sin(\alpha) \sin(\beta)$$
$$\cos(\alpha - \beta) = \cos(\alpha) \cos(\beta) + \sin(\alpha) \sin(\beta)$$
$$\tan(\alpha + \beta) = \frac{\tan(\alpha) + \tan(\beta)}{1 - \tan(\alpha)\tan(\beta)}$$
$$\tan(\alpha - \beta) = \frac{\tan(\alpha) - \tan(\beta)}{1 + \tan(\alpha)\tan(\beta)}$$
\subsection{Duplicazione}
$$\sin(2\alpha) = 2\sin(\alpha)\cos(\alpha)$$
$$\cos(2\alpha) = \cos^2 (\alpha) - \sin^2 (\alpha)$$
$$\tan(2\alpha) = \frac{\tan(\alpha)}{1 - \tan^2(\alpha)}$$
$$\cot(2\alpha) = \frac{\cot^2(\alpha) - 1}{2\cot(\alpha)}$$
\subsection{Bisezione}
$$\sin\left(\frac{\alpha}{2}\right) = \pm \sqrt{\frac{1 - \cos(\alpha)}{2}}$$
$$\cos\left(\frac{\alpha}{2}\right) = \pm \sqrt{\frac{1 + \cos(\alpha)}{2}}$$
$$\tan\left(\frac{\alpha}{2}\right) = {\frac{1 - \cos(\alpha)}{\sin(\alpha)}}$$
$$\cot\left(\frac{\alpha}{2}\right) = {\frac{1 + \cos(\alpha)}{\sin(\alpha)}}$$
\subsection{Werner}
$$\sin(\alpha)\sin(\beta) = \frac{1}{2} \left[\cos(\alpha - \beta) - \cos(\alpha + \beta)\right]$$
$$\cos(\alpha)\cos(\beta) = \frac{1}{2} \left[\cos(\alpha - \beta) + \cos(\alpha + \beta)\right]$$
$$\sin(\alpha)\cos(\beta) = \frac{1}{2} \left[\sin(\alpha - \beta) + \sin(\alpha + \beta)\right]$$
\hrule

\section{Derivate dimenticabili}
$$ (\tan(x))' = 1 + \tan^2(x) = \sec^2(x)$$
$$ (\arcsin(x))' = \frac{1}{\sqrt{1 - x^2}}$$
$$ (\arccos(x))' = - \frac{1}{\sqrt{1 - x^2}}$$
$$ (\arctan(x))' = \frac{1}{1 + x^2}$$
$$ (\sec(x))' = \tan(x)\sec(x)$$
$$ (\csc(x))' = -\cot(x)\csc(x)$$
$$ (\cot^n(x))' = - n \cdot \csc(x) \cdot \sec(x) \cdot \cot^n (x)$$
$$ (a^x)' = a^x \ln(a)$$
$$ (\log_a~x )' = \frac{1}{x\ln(a)}$$
\hrule

\section{Roba iperbolica}
$$ \sinh(x) = \frac{e^x - e^{-x}}{2} $$
$$\cosh(x) = \frac{e^x + e^{-x}}{2}$$
$$ \cosh^2(x) - \sinh^2(x) = 1 $$
$$ (\cosh(x))' = \sinh(x)$$
$$ \quad (\sinh(x))' = \cosh(x) $$


\section{Disuguaglianze utili}
$$ \abs{xy} \leq \frac{1}{2}\left(x^2 + y^2\right) $$
\hrule

\section{Sviluppi in serie}
$$ e^x = 1 + x + \frac{x^2}{2!} + \frac{x^3}{3!} + o(x^3)$$
$$ \ln(1 + x) = x - \frac{x^2}{2} + \frac{x^3}{3} + o(x^3)$$
$$ \sin(x) = x - \frac{x^3}{3!} + \frac{x^5}{5!} + o(x^6)$$
$$ \cos(x) = 1 - \frac{x^2}{2!} + \frac{x^4}{4!} + o(x^5)$$
$$ \tan(x) = x + \frac{x^3}{3} + \frac{2x^5}{15} + o(x^6)$$
$$ \arctan(x) = x - \frac{x^3}{3} + \frac{x^5}{5} + o(x^6)$$
$$ \frac{1}{1-x} = 1 + x + x^2 + x^3 + o(x^3)$$
$$(1 + x) ^ \alpha = 1 + \alpha x + \frac{\alpha (\alpha - 1)x^2}{2!} + \dots $$
$$ + \frac{\alpha(\alpha - 1)\dots (\alpha - n - 1) x^n}{n!} + o(x^n) $$
\hrule

\section{Integrali cattivi}
$$ \int \sin^2(x) \dif x = \frac{1}{2} (x - \sin(x) \cos(x)) $$
$$ \int \sin^3(x) \dif x = \frac{1}{12} (\cos(3x)-9\cos(x)) $$
$$ \int \cos^2(x) \dif x = \frac{1}{2} (x + \sin(x)\cos(x)) $$
$$ \int \cos^3(x) \dif x = \frac{1}{12} (\sin(3x) + 9\sin(x))$$
$$ \int \sin(x) \cos(x) \dif x = -\frac{1}{2} \cos^2(x) $$ 
$$ \int \tan(x) = -\ln(\cos(x)) $$
$$ \int \sec(x) \dif x = \ln (\tan(x) + \sec(x)) $$
$$ \int \sec^2(x) \dif x = \tan(x) $$
$$ \int \csc (x) \dif x = \ln \left( \tan\left(\frac{x}{2}\right)\right)$$
$$ \int \csc^2(x) \dif x = \cot(x) $$
$$ \int \arctan(x) \dif x = x \arctan(x) - \frac{1}{2} \ln (1 + x^2)$$
$$ \int \sqrt{a - x^2} \dif x = $$ $$ = \frac{x}{2} \sqrt{a - x^2} + \frac{a}{2} \arcsin \left(\frac {x}{a}\right) = $$
$$ =  \frac{x}{2} \sqrt{a - x^2} + \frac{a}{2} \arctan \left(\frac{x}{\sqrt{a-x^2}}\right) $$
$$ \int \sqrt{a + x^2} \dif x =  \frac{x}{2} \sqrt{a + x^2} + $$ 
$$ + \frac{a}{2} \ln \abs{x + \sqrt{a+x^2}} $$
$$ \int \frac{1}{\sqrt{x^2 + a^2}}\dif x = \log{\abs{\sqrt{x^2 + a^2} + x}}$$
$$ \int \frac{1}{\sqrt{a^2 - x^2}}\dif x = \cot \Big(\frac{x}{\sqrt{a-x^2}}\Big) = $$ $$ = \csc \Big( \frac{x}{\sqrt{a}}\Big)$$


\section{Successioni di funzioni}
Sia $\left\{ f_n(x)\right\}$ successione tendente puntualmente a $f(x)$ in $A\subset \mathbb{R}$
\begin{itemize}
    \item convergenza uniforme se $\sup_{x\in A}\left|f_n(x) - f(x)\right| \rightarrow 0$ per $n\rightarrow \infty$
    \item $f_n\rightrightarrows f \Leftrightarrow \left| f_m - f_n \right| \rightarrow 0$ ovvero Cauchy uniforme
    \item se $A$ compatto, $f_n$ continua e monotona, limite puntuale continuo, allora ho conv. unif.
    \item se le $f_n(x)$ sono continue ma $f(x)$ è discontinua, non posso avere conv. unif. 
    \item per verificare conv. unif. puoi maggiorare la succ. con una succ. numerica che va a 0
    \item se $f_n$ integrabili e $f_n\rightrightarrows f$, allora $\lim_{n\rightarrow \infty} \int_A f_n(x) \dif x = \int_A f \dif x$ (puoi usarlo al contrario eventualmente)
    \item se $f_n \in C^1$, $f_n\rightrightarrows f$ e $\frac{\dif f_n}{\dif x} \rightrightarrows g$, allora $\lim_{n\rightarrow \infty} \frac{\dif f_n}{\dif x} = \frac{\dif f}{\dif x}$ 
\end{itemize}
\hrule

\section{Serie numeriche}
Sia $\sum a_n$ con $a_n > 0$ una serie numerica
\begin{itemize}
    \item condizione necessaria: $a_n \rightarrow 0$ per $n \rightarrow \infty$
    \item criterio radice: sia $ \sqrt[n]{a_n} \rightarrow L$. Allora $L<1 \Rightarrow \sum a_n< \infty$, $L>1 \Rightarrow \sum a_n = + \infty$
    \item criterio rapporto: sia $\frac{a_n+1}{a_n} \rightarrow L$. Allora \\$L<1 \Rightarrow \sum a_n< \infty$, $L>1 \Rightarrow \sum a_n = + \infty$
    \item $\sum \left| a_n \right| < \infty \Rightarrow \sum a_n< \infty$
    \item criterio Leibniz: ho $\sum (-1)^n a_n$ con $a_n > 0$. Se $a_n \rightarrow 0$ e $a_n$ strett. decrescente allora ho convergenza
    \item $\sum \frac{1}{n^\alpha} < \infty$ per $\alpha > 1$
    
\end{itemize}
\hrule

\section{Serie di funzioni}
Sia $\sum_k f_k(x)$ serie di funzioni convergente in $A$
\begin{itemize}
    \item convergenza totale vuol dire $\sum_k \sup_{x \in A} \abs{f_k(x)} < \infty$
    \item test di Weiherstrass: se esiste successione reale $M_k$ t.c. $\left|f_k(x)\right| < M_k$ e $\sum M_k < \infty$ allora ho conv. tot.
\end{itemize}
\subsection{Serie di potenze}
Sia $\sum c_k x^k$ serie di potenze.
\begin{itemize}
    \item \textbf{Raggio di convergenza} $R \in [0, +\infty]$ tale che la serie \\- se $\abs{x} < R$ conv. tot. in tutti i compatti $\subset (-R, R)$\\- diverge per $\abs{x} > R$\\ - nulla può dirsi per $\abs{x} = R$
    \item \textbf{Teorema di Abel} \\Se c'è conv. punt. anche per $x = R$, allora conv. unif. in $[-a, R] \subset (-R, R]$. Stessa cosa per $x=-R$ ma col segno meno, dai hai capito.
    \item \textbf{Criterio di Cauchy-Hadamard} per trovare $R$: \\ $$R^{-1} = \limsup_{k\rightarrow \infty} \sqrt[k]{\abs{c_k}} $$ $$ = \limsup_{k \rightarrow \infty} \frac{\abs{c_{k+1}}}{\abs{c_k}}$$
    \item Serie derivate: sia $S(x) = \sum c_k x^k$. $S^n(x) = \sum (k+1)(k+2)\dots (k+n) c_{k+n} x^k$
\end{itemize}
\hrule

\section{Convergenza integrali impropri}
Le regole di confronto asintotico valgono solo per funzioni positive. Se ne hai una negativa prova con conv. assolusta.
\subsection{I specie}
Integrali di $f(x)$ limitata su intervalli illimitati, tipo $(- \infty, A]$, $[A, + \infty)$, $(-\infty, + \infty)$.
\begin{itemize}
    \item condizione necessaria $$\lim_ {x \rightarrow \infty} f(x) = 0$$
    \item convergenza assoluta implica non assoluta
    \item la funzione deve essere infinitesima rispetto a $1/x$, ovvero  all'infinito deve andare come $1/x^{\alpha}$ con $\alpha > 1$
    \item caso particolare con i logaritmi: $$ \int_{a>1}^{+ \infty} \frac{1}{x^\alpha \log^\beta (x)} \dif x  =$$ \\$$ = 
    \begin{cases*}
        < \infty \quad se~\alpha > 1, \forall \beta ~~ o ~~ \alpha = 1, \beta > 1 \\
        + \infty \quad se~\alpha < 1, \forall \beta ~~ o ~~ \alpha = 1, \beta \leq 1
    \end{cases*}$$
\end{itemize}
\subsection{II specie}
Integrali di $f(x)$ illimitate su intervalli limitati. 
\begin{itemize}
    \item la funzione deve essere infinita rispetto a $1/x$, ovvero deve andare come $1/x^{\alpha}$ con $\alpha < 1$.
    \item caso particolare con i logaritmi:
    $$ \int_0^{a<1} \frac{1}{x^\alpha \log^\beta (x)} \dif x  =$$ \\$$ = 
    \begin{cases*}
        < \infty \quad se~\alpha < 1, \forall \beta ~~ o ~~ \alpha = 1, \beta > 1 \\
        + \infty \quad se~\alpha > 1, \forall \beta ~~ o ~~ \alpha = 1, \beta \leq 1
    \end{cases*}$$
    \item attenzione quando verifichi gli $\alpha$, se trovi che la funzione non è nemmeno infinita va benissimo!
\end{itemize}
\hrule
\section{Integrazione secondo Lebesgue}
\begin{itemize}
    \item \textbf{Teorema convergenza monotona} \\ Sia $\{ f_n(x) \} \in L^1$ successione \underline{monotona q.o.} e sia $$\abs{\int f_n \dif x} \leq M~\forall n \in \mathbb{N}$$ allora $ \exists f \in L^1$ tale che $f_n \rightarrow f~ in~ L^1~e~f\rightarrow f~q.o.$
    \item \textbf{Teorema convergenza dominata}\\ Abbiamo una successione di funzioni \underline{integrabili} tali che $$ L^1 \ni f_n(x) \rightarrow f(x)~q.o.$$ e dominate da una funzione integrabile, ovvero $\exists~g(x)$ tale che $$ \abs{f_n(x)} \leq g(x) \in L^1~\forall x$$ allora $\Rightarrow f(x) \in L^1 ~ e ~ f_n(x) \rightarrow f~in~L^1$
    \item \textbf{Teorema di Fatou}\\ Abbiamo una successione di funzioni \underline{integrabili non negative} $L^1 \ni f_n(x) \geq 0$ e tali che $$f_n(x) \rightarrow f(x)~q.o.$$ $$\int f_n(x) \dif x \leq M~\forall n\in \mathbb{N}$$ allora $$ \Rightarrow f\in L^1~e~\int f(x) \dif x \leq M$$ 
    \item \textbf{Da $\mathbf{L^1_{loc}}$ a $\mathbf{L^1}$}\\Sia $f\in L^1_{loc}$ e $g\in L^1$, se $$\abs{f} \leq g \Rightarrow f\in L^1$$
    \item Def: \textbf{Funzione misurabile} \\ Una funzione $f:\mathbb{R}^n \rightarrow \mathbb{R}$ è misurabile se $$\forall I\subset \mathbb{R}~mis.~ \Rightarrow f^{-1}(I)~mis.$$ 
    \item \textbf{Misurabilià e integrabilità}\\ Sia $f$ misurabile; se esiste $g\in L^1_{loc}$ tale che $\abs{f} \leq g$, allora $f\in L^1_{loc}$
    \item \textbf{Fubini}\\ Sia $f(x) \in L^1(\mathbb{R}^n)$ e vedo $\mathbb{R}^n = \mathbb{R}^{d_1} \times \mathbb{R}^{d_1}$ Allora gli integrali parziali di $f$ rispetto a un insieme sono definiti quasi ovunque e ottengo l'integrale di $f$ in $\mathbb{R}^{n}$ facendo l'integrale iterato, che deve risultare uguale in tutti gli ordini possibili.
    \item \textbf{Tonelli}\\ Sia $f\in \mathbb{R}^n$ \underline{misurabile} e sia \underline{almeno uno} di $$ \int \limits_{\mathbb{R}^{d_2}} \dif y \int \limits_{\mathbb{R}^{d_1}} \abs{f(x, y)} \dif x \leq \infty $$ $$ \text{o quello inverito}~\leq \infty$$ allora $$f\in L^1$$ e ne faccio l'integrale iterato secondo Fubini.
\end{itemize}
\hrule
\section{Serie di Fourier}
Sia $f$ una funzione periodica nell'intervallo $[x_1, x_2]$, con periodo $T = x_2 - x_1$ e pulsazione $\omega = \frac{2\pi}{T}$. La sua serie di Fourier è $$ \frac{a_0}{2} + \sum_{k=1}^{\infty} \Big (a_k \cos(k\omega x) + b_l \sin(k\omega x)\Big)$$ 
con 
$$a_0 = \frac 2 T \int_{x_1}^{x_2} f(x) \dif x$$
$$ a_k = \frac{2}{T} \int_{x_1}^{x_2} f(x) \cos(k\omega x) \dif x$$ $$ b_k = \frac{2}{T} \int_{x_1}^{x_2} f(x) \sin(k\omega x) \dif x$$ \\
In forma complessa si scrive: $$ \sum_{k=-\infty}^{+\infty} c_k e^{ik\omega x}, ~ c_k = \frac 1 T \int_{x_1}^{x_2} f(x) e^{ik\omega x} \dif x$$
\begin{itemize}
    \item Se $f$ è \underline{pari}, allora tutti i  $b_k = 0$\\Se $f$ è \underline{dispari}, allora tutti gli $a_k = 0$.
    \item \textbf{Convergenza puntuale} della serie se $f \in L^1$ sull'insieme dove è periodica e vale almeno una delle due:\\ - $\exists M> 0~t.c.~ \abs{f(x+c) - f(x)} \geq M\abs{c}$ all'interno di un intorno $(x - \delta, x + \delta)$\\ - $\frac{f(x+c) - f(x)}{c} \in L^1$ nell'intervallo di periodicità.\\
    Se la funzione è regolare a tratti, ovvero derivabile a tratti, c'è convergenza puntuale (ma è un'ipotesi più forte)\\
    Attenzione che se c'è un salto la serie converge al \underline{valore a metà} $\frac{f(x^-) + f(x^+)}{2}$
    \item \textbf{Convergenza uniforme} della serie se $f\in C^1$ a tratti nell'intervallo di periodicità. Però $f$ è discontinua non posso avere conv. uniforme.
    \item \textbf{Convergenza in $\mathbf{L^2}$} se $f \in L^2$. In tal caso vale la \underline{uguaglianza di Parseval} $$ \frac 2 T \int_{x_1}^{x_2} \abs{f(x)}^2 \dif x = $$ $$ = \frac{\abs{a_0}^2}{2} +  \sum_{k=1}^{\infty} \Big( \abs{a_k}^2 + \abs{b_k}^2 \Big)$$
    $$ \frac 2 T \int_{x_1}^{x_2} \abs{f(x)}^2 \dif x  = \sum_{k=-\infty}^{+\infty} \abs{c_k}^2$$
\end{itemize}
\end{document}