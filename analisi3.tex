\documentclass[a4paper,portrait,columns=3,5pt]{cheatsheet}
\title{Formulario Analisi 3}
%\author{Elena Acinapura}
\author{}
\date{\today}

\usepackage{commath}
\usepackage{graphicx}
\usepackage{mathtools}
\usepackage{commath}
\usepackage{amssymb}
\usepackage{amsmath}
\usepackage{bigints}
\usepackage[utf8]{inputenc}

\begin{document}
%\maketitle

\section{Formule trigonometriche}
\subsection{Addizione e sottrazione}
$$\sin (\alpha + \beta) = \sin(\alpha) \cos(\beta) + \cos(\alpha)\sin(\beta)$$
$$\sin(\alpha - \beta) = \sin(\alpha) \cos(\beta) - \cos(\alpha)\sin(\beta)$$
$$\cos(\alpha + \beta) = \cos(\alpha) \cos(\beta) - \sin(\alpha) \sin(\beta)$$
$$\cos(\alpha - \beta) = \cos(\alpha) \cos(\beta) + \sin(\alpha) \sin(\beta)$$
$$\tan(\alpha + \beta) = \frac{\tan(\alpha) + \tan(\beta)}{1 - \tan(\alpha)\tan(\beta)}$$
$$\tan(\alpha - \beta) = \frac{\tan(\alpha) - \tan(\beta)}{1 + \tan(\alpha)\tan(\beta)}$$
\subsection{Duplicazione}
$$\sin(2\alpha) = 2\sin(\alpha)\cos(\alpha)$$
$$\cos(2\alpha) = \cos^2 (\alpha) - \sin^2 (\alpha)$$
$$\tan(2\alpha) = \frac{\tan(\alpha)}{1 - \tan^2(\alpha)}$$
$$\cot(2\alpha) = \frac{\cot^2(\alpha) - 1}{2\cot(\alpha)}$$
\subsection{Bisezione}
$$\sin\left(\frac{\alpha}{2}\right) = \pm \sqrt{\frac{1 - \cos(\alpha)}{2}}$$
$$\cos\left(\frac{\alpha}{2}\right) = \pm \sqrt{\frac{1 + \cos(\alpha)}{2}}$$
$$\tan\left(\frac{\alpha}{2}\right) = {\frac{1 - \cos(\alpha)}{\sin(\alpha)}}$$
$$\cot\left(\frac{\alpha}{2}\right) = {\frac{1 + \cos(\alpha)}{\sin(\alpha)}}$$
\subsection{Werner}
$$\sin(\alpha)\sin(\beta) = \frac{1}{2} \left[\cos(\alpha - \beta) - \cos(\alpha + \beta)\right]$$
$$\cos(\alpha)\cos(\beta) = \frac{1}{2} \left[\cos(\alpha - \beta) + \cos(\alpha + \beta)\right]$$
$$\sin(\alpha)\cos(\beta) = \frac{1}{2} \left[\sin(\alpha - \beta) + \sin(\alpha + \beta)\right]$$

\section{Derivate fondamentali}
$$ (\tan(x))' = 1 + \tan^2(x) = \sec^2(x)$$
$$ (\arcsin(x))' = \frac{1}{\sqrt{1 - x^2}}$$
$$ (\arccos(x))' = - \frac{1}{\sqrt{1 - x^2}}$$
$$ (\arctan(x))' = \frac{1}{1 + x^2}$$
$$ (\sec(x))' = \tan(x)\sec(x)$$
$$ (\csc(x))' = -\cot(x)\csc(x)$$
$$ (\cot^n(x))' = - n \cdot \csc(x) \cdot \sec(x) \cdot \cot^n (x)$$
$$ (a^x)' = a^x \ln(a)$$
$$ (\log_a~x )' = \frac{1}{x\ln(a)}$$

\section{Roba iperbolica}
$$ \sinh(x) = \frac{e^x - e^{-x}}{2} $$
$$\cosh(x) = \frac{e^x + e^{-x}}{2}$$
$$ \cosh^2(x) - \sinh^2(x) = 1 $$
$$ (\cosh(x))' = \sinh(x)$$
$$ \quad (\sinh(x))' = \cosh(x) $$

\section{Disuguaglianze utili}
$$ \abs{xy} \leq \frac{1}{2}\left(x^2 + y^2\right) $$

\section{Sviluppi in serie}
$$ e^x = 1 + x + \frac{x^2}{2!} + \frac{x^3}{3!} + o(x^3)$$
$$ \ln(1 + x) = x - \frac{x^2}{2} + \frac{x^3}{3} + o(x^3)$$
$$ \sin(x) = x - \frac{x^3}{3!} + \frac{x^5}{5!} + o(x^6)$$
$$ \cos(x) = 1 - \frac{x^2}{2!} + \frac{x^4}{4!} + o(x^5)$$
$$ \tan(x) = x + \frac{x^3}{3} + \frac{2x^5}{15} + o(x^6)$$
$$ \arctan(x) = x - \frac{x^3}{3} + \frac{x^5}{5} + o(x^6)$$
$$ \frac{1}{1-x} = 1 + x + x^2 + x^3 + o(x^3)$$
$$(1 + x) ^ \alpha = 1 + \alpha x + \frac{\alpha (\alpha - 1)x^2}{2!} + \dots $$
$$ + \frac{\alpha(\alpha - 1)\dots (\alpha - n - 1) x^n}{n!} + o(x^n) $$

\section{Successioni di funzioni}
Sia $\left\{ f_n(x)\right\}$ successione tendente puntualmente a $f(x)$ in $A\subset \mathbb{R}$
\begin{itemize}
    \item convergenza uniforme se $\sup_{x\in A}\left|f_n(x) - f(x)\right| \rightarrow 0$ per $n\rightarrow \infty$
    \item $f_n\rightrightarrows f \Leftrightarrow \left| f_m - f_n \right| \rightarrow 0$ ovvero Cauchy uniforme
    \item se $A$ compatto, $f_n$ continua e monotona, limite puntuale continuo, allora ho conv. unif.
    \item se le $f_n(x)$ sono continue ma $f(x)$ è discontinua, non posso avere conv. unif. 
    \item per verificare conv. unif. puoi maggiorare la succ. con una succ. numerica che va a 0
    \item se $f_n$ integrabili e $f_n\rightrightarrows f$, allora $\lim_{n\rightarrow \infty} \int_A f_n(x) \dif x = \int_A f \dif x$ (puoi usarlo al contrario eventualmente)
    \item se $f_n \in C^1$, $f_n\rightrightarrows f$ e $\frac{\dif f_n}{\dif x} \rightrightarrows g$, allora $\lim_{n\rightarrow \infty} \frac{\dif f_n}{\dif x} = \frac{\dif f}{\dif x}$ 
\end{itemize}

\section{Serie numeriche}
Sia $\sum a_n$ con $a_n > 0$ una serie numerica
\begin{itemize}
    \item condizione necessaria: $a_n \rightarrow 0$ per $n \rightarrow \infty$
    \item criterio radice: sia $ \sqrt[n]{a_n} \rightarrow L$. Allora $L<1 \Rightarrow \sum a_n< \infty$, $L>1 \Rightarrow \sum a_n = + \infty$
    \item criterio rapporto: sia $\frac{a_n+1}{a_n} \rightarrow L$. Allora \\$L<1 \Rightarrow \sum a_n< \infty$, $L>1 \Rightarrow \sum a_n = + \infty$
    \item $\sum \left| a_n \right| < \infty \Rightarrow \sum a_n< \infty$
    \item criterio Leibniz: ho $\sum (-1)^n a_n$ con $a_n > 0$. Se $a_n \rightarrow 0$ e $a_n$ strett. decrescente allora ho convergenza
    \item $\sum \frac{1}{n^\alpha} < \infty$ per $\alpha > 1$
    
\end{itemize}

\section{Serie di funzioni}
Sia $\sum_k f_k(x)$ serie di funzioni convergente in $A$
\begin{itemize}
    \item convergenza totale vuol dire $\sum_k \sup_{x \in A} \left|f_k(x)\right| < \infty$
    \item test di Weiherstrass: se esiste successione reale $M_k$ t.c. $\left|f_k(x)\right| < M_k$ e $\sum M_k < \infty$ allora ho conv. tot.
\end{itemize}
\subsection{Serie di potenze}

\end{document}